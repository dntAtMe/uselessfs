\chapter{Projekt systemu}
\thispagestyle{chapterBeginStyle}
W tym rozdziale przedstawiono scenariusze oraz przypadki użycia systemu. Na końcu znajduje się omówienie i przedstawienie w postaci pseudokodu niektórych algorytmów zastosowanych w implementacji. 

\section{Założenia i ograniczenia}
Na poziomie projektu nałożono ograniczenia na zakres działania omawianego systemu plików. Jedynie dwa rodzaje plików są obsługiwane: Pliki regularne (tekstowe) oraz katalogi. Wynika to z trudności implementacji, na przykład, połączeń symbolicznych, jeśli pliki są podzielone na kilka części. Dodatkowo założono, że na plikach dokonuje się tylko podstawowych operacji odczytu, zapisu, tworzenia oraz usuwania.

\section{Przypadki użycia i scenariusze}
\subsection{Montowanie systemu plików}
\begin{table}[h!]
        \centering
        \begin{tabular}{ |l|p{10cm}| }
                \hline
            S01 & Montowanie systemu plików \\ \hline
            Aktor & Użytkownik \\ \hline
            Warunki wsępne & Aktor przygotował konfigurację systemu \\ \hline
            & \\ Przebieg wydarzeń & \textbullet Aktor montuje system, podając plik konfiguracyjny \newline \newline 
            \textbullet System odczytuje konfigurację \newline \newline 
            \textbullet System dostosowuje parametry i montuje system plików \\
            & \\ \hline
            Alternatywny przebieg wydarzeń & \textbullet Aktor konfiguruje system przez argumenty wywołania \\ \hline
            Sytuacje wyjątkowe & \textbullet Aktor podał niepoprawną konfigurację \newline \newline
            \textbullet Katalogi podane przez aktora nie istnieją lub są niepoprawne \\ \hline
            Warunki końcowe & System plików jest poprawnie zamontowany pod podanym katalogiem \\ \hline
        \end{tabular}
        \caption{Montowanie systemu plików}
\end{table}
\newpage

\subsection{Odczyt pliku}

\begin{table}[h!]
        \centering
        \begin{tabular}{ |l|p{10cm}| }
                \hline
            S02.1 & Odczyt pliku w standardowej replice\\ \hline
            Aktor & Użytkownik \\ \hline
            Warunki wstępne & System plików jest zamontowany, tylko jedna aktywna replika\\ \hline
            Przebieg wydarzeń & 
            1. Aktor podaje plik do odczytu \newline \newline
            2. System otwiera plik, jesli nie ma do niego uchwytu  \newline \newline
            3. System odczytuje zadaną zawartość pliku \newline \newline
            4. System odczytuje nadmiarowe bity \newline \newline
            5. Na podstawie posiadanych informacji system ocenia integralność odczytanych danych \newline \newline
            6. System informuje aktora o uszkodzonej zawartości \newline \newline
            7. Aktor dostaje odczytane dane \\ \hline
            Alternatywny przebieg wydarzeń & 
            Brak \\ \hline
            Sytuacje wyjątkowe & \textbullet Błąd podczas otwierania pliku\newline \newline
            \textbullet Błąd oczytu \newline \newline
            \textbullet Dysk z repliką został odmontowany lub uszkodzony \\ \hline
            Warunki końcowe & System zwrócił aktorowi odczytane dane \\ \hline
        \end{tabular}
        \caption{Odczyt pliku w standardowej replice}
\end{table}
\newpage
\begin{table}[h!]
        \centering
        \begin{tabular}{ |l|p{10cm}| }
                \hline
            S02.2 & Odczyt pliku w replice korekcyjnej\\ \hline
            Aktor & Użytkownik \\ \hline
            Warunki wstępne & System plików jest zamontowany, replika wspiera korekcję błędów \\ \hline
            Przebieg wydarzeń & 
            1. Aktor podaje plik do odczytu \newline \newline
            2. System otwiera plik, jeśli nie ma do niego uchwytu  \newline \newline
            3. System odczytuje zadaną zawartość pliku \newline \newline
            4. System odczytuje nadmiarowe bity \newline \newline
            5. Na podstawie posiadanych informacji system ocenia integralność odczytanych danych \newline \newline
            6. System podejmuje próbę naprawy znalezionych błędów \newline \newline
            6. System informuje aktora o nadal uszkodzonej zawartości \newline \newline
            7. Aktor dostaje odczytane dane \\ \hline
            Alternatywny przebieg wydarzeń & 
            5. Nie znaleziono błędów\newline \newline
            6. Aktor dostaje odczytane dane\\ \hline
            Sytuacje wyjątkowe & \textbullet Błąd podczas otwierania pliku\newline \newline
            \textbullet Nieudany odczyt pliku \newline \newline
            \textbullet Dysk z repliką został odmontowany lub uszkodzony \\ \hline
            Warunki końcowe & System zwrócił aktorowi odczytane dane \\ \hline
        \end{tabular}
        \caption{Odczyt pliku w replice korekcyjnej}
\end{table}

\newpage
\begin{table}[h!]
        \centering
        \begin{tabular}{ |l|p{10cm}| }
                \hline
            S02.4 & Odczyt pliku przy wykorzystaniu wielu replik\\ \hline
            Aktor & Użytkownik \\ \hline
            Warunki wstępne & System plików jest zamontowany, wiele replik aktywnych \\ \hline
            Przebieg wydarzeń & 
            1. Aktor podaje plik do odczytu \newline \newline 
            2. System dokonuje wyboru repliki\newline \newline
            3. System przekazuje replice plik do odczytu\newline \newline
            4. System naprawia błędy przenosząc dane z kolejnej najlepszej repliki \newline \newline
            5. Aktor dostaje odczytane dane \\ \hline
            Alternatywny przebieg wydarzeń & 
            4. Dane w replice nie zostały naprawione \newline \newline
            5. System wyłącza replikę \newline \newline \newline \newline
            4. Nie ma wiecej replik \newline \newline
            5. System informuje aktora o bledach \newline \newline
            6. Aktor dostaje odczytane dane\\ \hline
            Sytuacje wyjątkowe & \textbullet Blad podczas otwierania pliku\newline \newline
            \textbullet Błąd oczytu \newline \newline
            \textbullet Replika jest jedyną działającą \\ \hline
            Warunki końcowe & System zwrócił aktorowi odczytane dane i replika została naprawiona lub wyłączona \\ \hline
        \end{tabular}
        \caption{Odczyt pliku przy wykorzystaniu wielu replik}
\end{table}
\newpage
\begin{table}[h!]
        \centering
        \begin{tabular}{ |l|p{10cm}| }
                \hline
            S02.5 & Odczyt brakujacego pliku w replice\\ \hline
            Aktor & Użytkownik \\ \hline
            Warunki wstępne & System plików jest zamontowany \\ \hline
            Przebieg wydarzeń & 
            1. Aktor podaje plik do odczytu \newline \newline 
            2. System wybiera najlepszą replikę\newline \newline
            3. System wykrywa brak pliku w replice\newline \newline
            4. System sprawdza obecność nieuszkodzonego pliku w pozostałych replikach \newline \newline
            6. Brakujące dane zostają zsynchronizowane między replikami\\ \hline
            Alternatywny przebieg wydarzeń & 
            \textbullet Brak innych replik \newline \newline
            \textbullet Brak pliku we wszystkich replikach \newline \newline
            \textbullet Znalezione kopie pliku są uszkodzone, występuje konflikt \\ \hline
            Sytuacje wyjątkowe &
            \textbullet Błąd podczas otwierania pliku \newline \newline
            \textbullet Błąd odczytu \\ \hline
            Warunki końcowe & System zwrócił aktorowi odczytane dane i dane zostały zsynchronizowane lub aktor został poinformowany o braku pliku\\ \hline
        \end{tabular}
        \caption{Odczyt brakującego pliku w replice}
\end{table}

\newpage
\subsection{Zapis pliku}
\begin{table}[h!]
        \centering
        \begin{tabular}{ |l|p{10cm}| }
                \hline
            S03.1 & Zapis pliku przy wykorzystaniu pojedynczej repliki korekcyjnej \\ \hline
            Aktor & Użytkownik \\ \hline
            Warunki wstępne & System plików jest zamontowany, replika wspiera korekcję błędów \\ \hline
            Przebieg wydarzeń & 
            1. Aktor podaje dane do zapisu \newline \newline 
            2. System dopisuje dodatkowe informacje do korekty \newline \newline 
            3. System zapisuje do repliki \\ \hline
            Alternatywny przebieg wydarzeń &
            4. System nie może dopisać dodatkowych informacji  \newline \newline
            5. System zwraca błąd zapisu \\ \hline
            Sytuacje wyjątkowe & 
            \textbullet Nieudany zapis pliku \newline \newline
            \textbullet Dysk z repliką został odmontowany lub uszkodzony \newline \newline
            \textbullet Brak miejsca na replice \\ \hline
            Warunki końcowe & System zapisał dane podane przez aktora oraz informacje do korekcji \\ \hline
        \end{tabular}
        \caption{Zapis pliku przy wykorzystaniu pojedynczej repliki korekcyjnej} 
\end{table}

\begin{table}[h!]
        \centering
        \begin{tabular}{ |l|p{10cm}| }
                \hline
                S03.2 & Zapis pliku przy wykorzystaniu pojedynczej repliki standardowej\\ \hline
            Aktor & Użytkownik \\ \hline
            Warunki wstępne & System plików jest zamontowany, replika nie wspiera korekcji błędów \\ \hline
            Przebieg wydarzeń & 
            1. Aktor podaje dane do zapisu \newline \newline
            2. System dopisuje dodatkowe informacje do detekcji błędow \newline \newline
            2. System zapisuje do repliki \\ \hline
            Alternatywny przebieg wydarzeń &
            3. System nie może zapisać do repliki  \newline \newline
            4. System zwraca błąd zapisu \\ \hline
            Sytuacje wyjątkowe & 
            \textbullet Nieudany zapis pliku \newline \newline
            \textbullet Dysk z repliką został odmontowany lub uszkodzony \newline \newline
            \textbullet Brak miejsca na replice \\ \hline
            Warunki końcowe & System zapisał dane podane przez aktora \\ \hline
        \end{tabular}
        \caption{Zapis pliku przy wykorzystaniu pojedynczej repliki standardowej}
\end{table}
\newpage

\begin{table}[h!]
        \centering
        \begin{tabular}{ |l|p{10cm}| }
                \hline
            S03.3 & Zapis pliku przy wykorzystaniu wielu replik\\ \hline
            Aktor & Użytkownik \\ \hline
            Warunki wstępne & System plików jest zamontowany, wiele replik \\ \hline
            Przebieg wydarzeń & 
            1. Aktor podaje dane do zapisu \newline \newline 
            2. System zapisuje dane do każdej repliki \newline \newline 
            3. Każdy błąd zapisu wyłącza daną replikę\\ \hline
            Alternatywny przebieg wydarzeń &
            Brak \\ \hline
            Sytuacje wyjątkowe & 
            \textbullet Nieudany zapis pliku do każdej repliki\newline \newline
            \textbullet Wszystkie repliki zostały wyłączone \\ \hline
            Warunki końcowe & System zapisał dane podane przez aktora i są działające repliki\\ \hline
        \end{tabular}
        \caption{Zapis pliku przy wykorzystaniu wielu replik} 
\end{table}

\begin{table}[h!]
        \centering
        \begin{tabular}{ |l|p{10cm}| }
                \hline
            S04.1 & Wybór najlepszej repliki\\ \hline
            Aktor & Użytkownik \\ \hline
            Warunki wstępne & System plików jest zamontowany, przynajmniej jedna replika\\ \hline
            Przebieg wydarzeń & 
            1. System sprawdza czy są aktywne repliki \newline \newline 
            2. System wybiera najlepszą replikę z aktywnych \\ \hline
            Alternatywny przebieg wydarzeń &
            Brak \\ \hline
            Sytuacje wyjątkowe & 
            \textbullet Wszystkie repliki zostały wyłączone \\ \hline
            Warunki końcowe & System wybrał replikę i wykonuje operacje aktora\\ \hline
        \end{tabular}
        \caption{Wybór najlepszej repliki} 
\end{table}

\newpage
\section{Opis algorytmów}
W implementacji wykorzystano algorytmy odtwarzające bity parzystości i kod Hamminga, jednak najważniejszą funkcją jest podział zadań pomiędzy różnymi replikami. Wiele z nich wykorzystuje wspólne funkcje. Głównym trzonem całego programu jest odczyt oraz zapis plików.
\\
\\
\\
{\small
\begin{pseudokod}[H]
%\SetAlTitleFnt{small}
\KwIn{Filepath}
\KwOut{Errorcode, Buffer}
    Errorcodes = [0]\;
\ForEach{$replica, i \in replicas, replicas.size$}{
    \If{$replica = $BLOCK}{
        Errorcodes[i] = BlockReplicaRead(path, replica, Buffer)\;
    }
    \If{$replica = $MIRROR}{
        Errorcodes[i] = MirrorReplicaRead(path, replica, Buffer)\;
    }
    \If{$Errorcodes[i] = 0$}{
        break;
    }
    HandleError(Errorcodes[i], replica)\;
}
    HandleErrors(Errorcodes, i)\;
Return;
\caption{Odczyt pliku}\label{alg:1}
\end{pseudokod}
}

Algorytm \ref{alg:1} dzieli funkcjonalność systemu w zależności od rodzaju repliki. W ten sposób użytkownik może zaimplementować własny rodzaj repliki, dokonując tylko niewielkich zmian w już istniejącym kodzie. W linii 9 znajduje się obsługa błędów, część wspólna dla wszystkich rodzajów replik. Określa zachowanie programu w przypadku, kiedy replika ma uszkodzone dane, których nie jest w stanie naprawić, lub jeśli wystąpiły błędy podczas odczytu pliku, takie jak brakujący plik. Jeśli naprawa wymaga repliki, w której odczyt zakończył się sukcesem, takie repliki są naprawianie w linii 10.
\\
\\
\\
{\small
\begin{pseudokod}[H]
%\SetAlTitleFnt{small}
\KwIn{Filepath, Replica}
\KwOut{Errorcode, Buffer}
\ForEach{$block \in Replica.blocks$}{
        Errorcode = ReadFile( $block.path$ + Filepath, Buffer)\;
        \If{$ErrorCode$}{
            Return;
        }
        \If{$ShouldInterlaceRedundancy(Replica)$}{
            Checksum = GetChecksum(Buffer)\;
            ParityBuffer = GetOddBytes(Buffer)\;
            Buffer = GetEvenBytes(Buffer)\;
            CalculatedParityBuffer = CalculateParity(Buffer, Replica)\;
            \If{CalculatedParityBuffer != ParityBuffer}{
                \If{$ShouldCorrectErrors(Replica)$}{
                    CorrectErrors(Buffer, block)\;
                    CalculateChecksum(block)\;
                    Return\;
                }
                Return;
            }
            CalculatedChecksum = CalculateChecksum(block)\;
            \If{CalculatedChecksum != Checksum}{
                Return;
            }
        }
    }
    Return;
\caption{Odczyt pliku z repliki blokowej}\label{alg:2}
\end{pseudokod}
}

Algorytm \ref{alg:2} dokonuje odczytu zawartości pliku w replice. W zależności od konfiguracji repliki, wykonywane są różne czynności. Jeśli warunek w linii 3. jest spełniony,to w trakcie próby odczytu z bloku wystąpił błąd. Spełnienie warunku w linii 5. oznacza, że w replice jest stosowane przeplatanie danych z redundantnymi bitami. Funkcja w linii 7 oblicza redundantne bity, czyli kod Hamminga lub bit parzystości, w zależności od konfiguracji. Następnie obliczony kod z odczytanego buforu jest porównywany z kodem dołączonym do odczytywanych danych. Replika podejmuje się próby naprawy błędów, jeśli jest taka możliwość. Jeśli nie znaleziono błędów, to możliwe, że dane zostały uszkodzone, ale użyty kod nie wykrył błędów, więc dla pewności replika sprawdza sumę kontrolną pliku.
\\
\\
\\
\\
{\small
\begin{pseudokod}[H]
%%\SetAlTitleFnt{small}
\KwIn{Buffer, Path}
\KwOut{Errorcode}
    \ForEach{$replica \in replicas$}{
        FileHandlers = GetOpenedFileHandlers(replica, Path)\;
        \If{$replica = $BLOCK}{
            Errorcode = BlockReplicaWrite(FileHandlers, replica, Buffer)\;
        }
        \If{$replica = $MIRROR}{
            Errorcode = MirrorReplicaRead(FileHandlers, replica, Buffer)\;
        }
        HandleError(Errorcode, replica)\;
   }
    Return;
\SetAlTitleFnt{small}
\caption{Zapis danych do pliku}\label{alg:3}
\end{pseudokod}
}
Algorytm zapisu \ref{alg:3} zachowuje się analogicznie do odczytu w \ref{alg:1}. Iteruje po wszystkich replikach i podejmuje się próby zapisu, naprawiając napotkane błędy. Główna różnica jest taka, że odczytać wystarczy jedną poprawną kopię, aby uzyskać dane z pliku. Zapis musi się odbyć do wszystkich replik.
\\
\\
\\
{\small
\begin{pseudokod}[H]
%%\SetAlTitleFnt{small}
\KwIn{Buffer, Replica, FileHandle}
\KwOut{Errorcode}
ParityBuffer = CalculateParity(Buffer, Replica)\;
\ForEach{$block \in Replica.blocks$}{
    Errorcode = WriteFile(FileHandle, Buffer, ParityBuffer, ShouldInterlaceRedundancy(Replica))\;
    \If{$Errorcode$}{
        Return;
    }
    Errorcode = AttachRedundancy(FileHandle, ParityBuffer, ShouldAttachRedundancy(Replica))\;
    \If{$Errorcode$}{
        Return;
    }   CalculatedChecksum = CalculateChecksum(block)\;
    AttachChecksum(FileHandle);
    
        }
    Return;
\SetAlTitleFnt{small}
\caption{Zapis danych do pliku w replice blokowej}\label{alg:4}
\end{pseudokod}
}

Algorytm \ref{alg:4} zapisuje dane do pliku, przeplatając i dołączając dodatkowe kody zależnie od swoich opcji. Pod koniec dołączana jest suma kontrolna na początek pliku i zapis kończy się sukcesem.
