\chapter{Instalacja i wdrożenie}
\thispagestyle{chapterBeginStyle}

Implementowany program został przetestowany na systemach Fedora 30 oraz Ubuntu 18.04 na systemie plików ext4. Moduł FUSE jest modułem Unixowych systemów operacyjnych i jest wymagany do działania projektu, potrzebna jest również biblioteka \verb|openssl|.

Zakłada się, że operacje wykonywane na plikach ograniczają się do odczytu oraz zapisu plików. Dane powinny być zapisywane bajtami, na przykład w charakterze liter alfabetu.

Kompilacja kodu źródłowego odbywa się przy wykorzystaniu programu \verb|make|, otrzymany plik służy do zamontowania systemu plików. Podczas montowanie systemu plików zaleca się skorzystanie z gotowej konfiguracji ustawiającej argumenty według wzoru:
\\
\\
\verb| --number ilość replik| \\
\verb| --tmp katalog zapasowy dla odzyskiwanych danych| \\

Dla każdej repliki należy dodać następujące argumenty:
\\
\verb| --block-replica ilość_bloków flagi <ścieżki do bloków>| \\
\verb| --mirror-replica flagi ścieżka_do_repliki | \\
\\

System plików można zamontować wywołując skompilowany plik w terminalu:
\\
 \verb|./uselessfs [punkt zamontowania] <lista argumentów / @plik>|
\\
\textbf{punkt zamontowania} - Katalog, który ma służyć użytkownikowi do interakcji z zamontowanym systemem plików.
\\
\textbf{lista argumentów}  - argumenty potrzebne do zamontowania plików, mogą być odczytane z pliku \\
Po wykonaniu powyższych czynności, można zacząć pracę z omawianym systemem plików.

System plików nie jest w pełni funkcjonalny, z tego powodu dołączono pliki konfiguracyjne zawierające przykładowe konfiguracje replik do uruchomienia.
Dodatkowo, do katalogu z programem zostały dodane przykładowe testy pokazujące możliwe zastosowania replik oraz sposoby na symulację uszkodzenia danych. Uruchomienie programu bez argumentów dostarcza informacji na temat systemu plików oraz stanu zaawansowania projektu.

Do kodu źródłowego dołączono skrypt \verb|prepare_env.sh|, przygotowujący środowisko do pracy z przykładowymi konfiguracjami.

