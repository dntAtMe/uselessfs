\chapter{Wstęp}
\thispagestyle{chapterBeginStyle}

Praca dyplomowa swoim zakresem obejmuje detekcję i korekcję błędów, niezawodność danych oraz wybrane metody redundancji. Celem pracy jest zaprojektowanie i implementacja systemu plików o następujących założeniach:
\begin{itemize}
    \item Detekcja błędów w plikach
    \item Korekcja znalezionych błędów
    \item Zapisywanie wielu kopii danych
    \item Odzyskiwanie utraconych danych 
\end{itemize}

Chociaż istnieją systemy plików umożliwiające zarówno odzyskiwanie, jak i zabezpieczanie danych przed uszkodzeniami, omawiany projekt w założeniu działa inaczej niż znalezione systemy. Przeważnie redundantne systemy plików są tworzone z myślą o grupach dysków, łącząc je we wspólną jednostkę logiczną. Zastosowane rozwiązania do odzyskiwania danych są często zakorzenione głęboko w logice takiego systemu, znacząco utrudniając wprowadzanie nowych rozwiązań.

Praca składa się z czterech rozdziałów. W rozdziale pierwszym poddano szczegółowej analizie metody detekcji oraz korekcji błędów, rozważono istniejące modele podziału oraz kopiowania danych i przedstawiono rozwiązania wykorzystane w implementowanym systemie.

W rozdziale drugim przedstawiono scenariusze do opisania ogólnego działania systemu oraz pseudokod algorytmów wykorzystanych w projekcie wraz z opisem zastosowanych rozwiązań.

W rozdziale trzecim opisano technologie implementacji projektu oraz strukturę kodu źródłowego. Przedstawiono zaimplementowe funkcjonalności oraz wynikające z nich ograniczenia.

W rozdziale czwartym przedstawiono sposób instalacji i zamontowania systemu w środowisku docelowym. Zostały omówione przykładowe metody użycia oraz załączone testy systemu.

