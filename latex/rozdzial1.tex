\chapter{Analiza problemu}
\thispagestyle{chapterBeginStyle}
\label{rozdzial1}

Analiza problemu
---------------------------------
W tym rozdziale należy przedstawić analizę zagadnienia, które podlega informatyzacji. Należy zidenty-
fikować i opisać obiekty składowe rozważanego wycinka rzeczywistości i ich wzajemne relacje (np. użytkown-
ików systemu i ich role). Należy szczegółowo omówić procesy jakie zachodzą w systemie i które będą infor-
matyzowane, takie jak np. przepływ dokumentów. Należy sprecyzować i wypunktować założenia funkcjon-
alne i poza funkcjonalne dla projektowanego systemu. Jeśli istnieją aplikacje realizujące dowolny podzbiór
zadanych funkcjonalności realizowanego systemu należy przeprowadzić ich analizę porównawczą, wskazując na
różnice bądź innowacyjne elementy, które projektowany w pracy system informatyczny będzie zawierał. Należy
odnieść się do uwarunkowań prawnych związanych z procesami przetwarzania danych w projektowanym sys-
temie. Jeśli zachodzi konieczność, należy wprowadzić i omówić model matematyczny elementów systemu na
odpowiednim poziomie abstrakcji.
{\color{dgray}
W niniejszym rozdziale omówiono koncepcję architektury programowej systemu \ldots. W
szczególny sposób \ldots. Omówiono założenia funkcjonalne i niefunkcjonalne podsystemów \ldots. Przedstawiono
mechanizmy \ldots. Sklasyfikowano systemy ze względu na \ldots. Omówiono istniejące rozwiązania informatyczne o podobnej funkcjonalności \ldots (zobacz \cite{JCINodesChord}).

-Omowienie sposobow redundancji danych -RAID -Istniejace rozwiazania -Wybrane rozwiazania
-Problem rozbijania danych -Problem wielu replik -Problem synchronizacji -Problem obsługi pełnej funkcjon-
alności, np. softlinks
Założenia: -Rodzaj wykorzystywanego sposobu redundacji niewidoczny dla uzytkownika -Uzytkownik nie
martwi sie replikami poza wyjatkowymi sytuacjami: synchronizacja replik przy zamontowaniu, zatwierdzenie
korekty
}

\begin{itemize}
    \item Obsługa kilku różnych rozwiązań redundancji
    \item Obsługa wielu replik dla większego bezpieczeństwa danych 
    \item Wybór najodpowiedniejszej repliki do odczytu danych
    \item Synchronizacja istniejących danych pomiędzy replikami przy zamontowaniu systemu plików
	\item Obsługa przynajmniej podstawowych zachowań systemu plików:
		\begin{itemize}
			\item Czytanie oraz pisanie do plików
			\item Tworzenie oraz usuwanie plików
		\end{itemize}
	\item Detekcja błędów:
		\begin{itemize}
			\item Sprawdzanie, czy pliki, na których dokonywane są operacje, nie są uszkodzone
			\item Sprawdzanie synchronizacji danych między replikami
		\end{itemize}
	\item Korekcja błędów:
		\begin{itemize}
			\item Zastosowanie kodów korekcyjnych
			\item Całkowite zastępowanie uszkodzonych danych replikami
		\end{itemize}
\end{itemize}


