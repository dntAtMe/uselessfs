Spis treści
1 Wstęp 1
2 Analiza problemu 3
3 Projekt systemu
3.1 Grupy użytkowników i założenia
3.2 Przypadki użycia i scenariusze .
3.3 Opis wykorzystanych rozwiązań .
3.4 Konfiguracja . . . . . . . . . . .
3.5 Gniazdo sterujące . . . . . . . . .
3.6 Diagramy . . . . . . . . . . . . .
3.7 Opis protokołów . . . . . . . . .
3.8 Opis algorytmów . . . . . . . . .
4 Implementacja systemu
4.1 Opis technologii . . . . . . . . . . .
4.2 Omówienie kodów źródłowych . . .
4.3 Omówienie pliku konfiguracyjnego
4.4 Omówienie wydajności systemu . . .
5.1 Kompilacja . . . . . . . .
5.2 Przykładowa konfiguracja
5.3 Gniazdo sterujące . . . . .
5.4 Uruchamianie . . . . . . . .
6 Podsumowanie 11
Bibliografia 13
A Zawartość płyty CD

Wstęp
Praca swoim zakresem obejmuje systemy plików, kodowanie
Celem pracy jest zaprojektowanie oraz implementacja wirtualnego systemu plików w przestrzeni użytkownika o następujących założeniach funkcjonalnych:
• ochrona w przypadku częściowego uszkodzenia danych
• redundacja danych (kopie plików, kopiowane sektory na wiele podsystemów, CRC + checksum)
• detekcja błędów
• korekcja błędów (kody korekcyjne)
• przetrzymywanie kopii danych na wielu dyskach
•
Istnieje szereg systemów plików, również wirtualnych, zapewniających bezpieczeństwo danych w postaci
redundancji, w większości rozproszonych. [1]
Zakres pracy
Cel pracy
Analiza istniejących rozwiązań, przegląd literatury
Opis zawartości pracy


Analiza problemu
---------------------------------
W tym rozdziale należy przedstawić analizę zagadnienia, które podlega informatyzacji. Należy zidenty-
fikować i opisać obiekty składowe rozważanego wycinka rzeczywistości i ich wzajemne relacje (np. użytkown-
ików systemu i ich role). Należy szczegółowo omówić procesy jakie zachodzą w systemie i które będą infor-
matyzowane, takie jak np. przepływ dokumentów. Należy sprecyzować i wypunktować założenia funkcjon-
alne i poza funkcjonalne dla projektowanego systemu. Jeśli istnieją aplikacje realizujące dowolny podzbiór
zadanych funkcjonalności realizowanego systemu należy przeprowadzić ich analizę porównawczą, wskazując na
różnice bądź innowacyjne elementy, które projektowany w pracy system informatyczny będzie zawierał. Należy
odnieść się do uwarunkowań prawnych związanych z procesami przetwarzania danych w projektowanym sys-
temie. Jeśli zachodzi konieczność, należy wprowadzić i omówić model matematyczny elementów systemu na
odpowiednim poziomie abstrakcji.
W niniejszym rozdziale omówiono koncepcję architektury programowej systemu . . . . W szczególny sposób
. . . . Omówiono założenia funkcjonalne i niefunkcjonalne podsystemów . . . . Przedstawiono mechanizmy . . . .
Sklasyfikowano systemy ze względu na . . . . Omówiono istniejące rozwiązania informatyczne o podobnej
funkcjonalności . . . (zobacz [2]).
--------------------------------------
-Omowienie sposobow redundancji danych -RAID -Istniejace rozwiazania -Wybrane rozwiazania
-Problem rozbijania danych -Problem wielu replik -Problem synchronizacji -Problem obsługi pełnej funkcjon-
alności, np. softlinks
Założenia: -Rodzaj wykorzystywanego sposobu redundacji niewidoczny dla uzytkownika -Uzytkownik nie
martwi sie replikami poza wyjatkowymi sytuacjami: synchronizacja replik przy zamontowaniu, zatwierdzenie
korekty

3.1 Grupy użytkowników i założenia
Założenia funkcjonalne i poza funkcjonalne, zastosowanie, redundacja danych, bezpiczeństwo prechowywania plików, kolejne funkcje jako warstwy.
3.2 Przypadki użycia i scenariusze
Opis możliwych scenariuszy utraty danych i w jaki sposób zostaną odzyskane; przypadki użycia nie tylko
redundacji, ale i wykorzystania kilku dysków.
3.3 Opis wykorzystanych rozwiązań
Opis kolejnych poziomów RAID, które stanowią podstawę w imlementacji systemu. RAID-0, RAID-
1, RAID-3, RAID-4, RAID-5, RAID-6 W jaki sposób system poszukuje najszybszego dysku Opis gniazda
unixowego Wirtualne systemu plików w kontekście danych; brak przechowywania, tylko translacja konsek-
wencje tego podejścia
3.4 Konfiguracja
Sposób opisania żądanych zachowań oraz konfiguracji replik; Przykładowo, mountpoint ma znajdować
się na dwóch replikach RAID-5 A i B, replika A ma mieć repliki RAID-1 C i D, replika B ma mieć repliki
RAID-0 E i F.

3.5 Gniazdo sterujące
Opis działania gniazda sterującego, które umożliwia symulowanie pewnych zachowań w systemie plików,
odczytywanie logów systemu plików oraz dynamiczne ustawianie parametrów
3.6 Diagramy
Komunikacja pomiędzy kolejnymi warstwami systemu
3.7 Opis protokołów
3.8 Opis algorytmów
Opis zastosowanych rozwiązań; reed-solomon, kolejne raidy, szukanie najlepszej repliki


Implementacja systemu
4.1 Opis technologii
FUSE, C, RAID, Unix (sockets, ext4 fs)
4.2 Omówienie kodów źródłowych
4.3 Omówienie pliku konfiguracyjnego
4.4 Omówienie wydajności systemu
Porównanie zastosowanych rozwiązań, szukając n̈ajbardziej optymalnego,̈ zbadanie narzutu systemu jako
nakładka na istniejący system plików, omówienie możliwych rozwiązań na obniżenie narzutu, przyśpieszenie działania


Kompilacja i uruchamianie
5.1 Kompilacja
Kompilacja jako ważny czyynik implementowanego systemu plików; brak potrzeby sprawdzania paramet-
rów dynamicznie przyśpieszy działanie
5.2 Przykładowa konfiguracja
Przedstawienie przykładowej konfiguracji
5.3 Gniazdo sterujące
Jak uruchomić oraz korzystać z gniazda sterującego
5.4 Uruchamianie
Montowanie systemu plików


Podsumowanie
W podsumowanie należy określić stan zakończonych prac projektowych i implementacyjnych. Zaznaczyć,
które z zakładanych funkcjonalności systemu udało się zrealizować. Omówić aspekty pielęgnacji systemu
w środowisku wdrożeniowym. Wskazać dalsze możliwe kierunki rozwoju systemu, np. dodawanie nowych
komponentów realizujących nowe funkcje.
W podsumowaniu należy podkreślić nowatorskie rozwiązania zastosowane w projekcie i implementacji
(niebanalne algorytmy, nowe technologie, itp.).
